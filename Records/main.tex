\documentclass[UTF8,oneside,12pt]{book}
% 各种设置都在下面的文件里
\input{setting.tex}

% pdf文件设置
\hypersetup{
	pdfauthor={},
	pdftitle={}
}

% 在这输入你的标题,会加入页眉
\newcommand\TheTitle{}

% Mac 字体设置
% \setmainfont{TimesNewRomanPSMT}
% \setsansfont{Helvetica-Light}
% \setCJKmainfont[ItalicFont=STKaitiSC-Regular,BoldFont=STSongti-SC-Black]{STSongti-SC-Regular}
% \setCJKsansfont[BoldFont=STHeitiSC-Medium]{STHeitiSC-Light}
% \setCJKmonofont{STKaitiSC-Bold}% 加粗楷体
% \newfontfamily\ktb{STKaitiSC-Bold}

\newtheorem{Definition}{\hspace{2em}定义}[section]
\newtheorem{Lemma}{\hspace{2em}引理}[section]
\newtheorem{Theorem}{\hspace{2em}定理}[section]
\newtheorem*{Proof}{\hspace{0em}证明}


\usepackage{enumitem}
\usepackage{bm}


\begin{document}
% 设置中文
\input{setc.tex}
\setlength{\baselineskip}{20pt} % 行间距



%一、Mac Terminal 的Linux语法记录
\chapter{Mac Terminal的Linux语法}

%二、Git基本语法介绍
\chapter{Git基本语法}




% 参考文献
%\newpage
\bibliographystyle{gbt7714-2005}
\clearpage
\phantomsection
\addcontentsline{toc}{chapter}{参考文献} %向目录中添加条目,以章的名义
\bibliography{thesis}



%\nocite{*} %不管是否引用都列出参考文献

\end{document}




