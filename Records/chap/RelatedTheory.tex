\chapter{相关理论简介}

\section{一维布朗运动}
\begin{Definition}[一维标准布朗运动]
	给定概率空间$(\Omega,\mathcal{F},P)$及其上的$\sigma$-域流$\mathbb{F}$,定义在该空间的满足以下条件的连续时间一维适应过程$(B_t)_{t \in R_+}$:
	\begin{enumerate}[fullwidth, itemindent = 0em]
		\item[1)] $B_0 = 0,a.s.;$
		\item[2)] $B$是独立增量过程,即对$\forall s \le t$,$B_t-B_s$与$B_u(u\le s)$独立;且$B_t-B_s$独立于$\mathcal{F}_s$;
		\item[3)] $B$具有平稳增量,即对$\forall s \le t$,$B_t-B_s\stackrel{d}{\sim}B_{t-s}$,且$B_t-B_s\sim N(0,t-s),0\le s \le t;$
	\end{enumerate}
	为一维标准布朗运动或维纳过程。
\end{Definition}

\begin{Definition}[算数布朗运动]
	若随机过程$\{X_t\}_{t\ge0}$可以表示为:
	\begin{equation}
		\label{BM1}
		dX_t =\mu dt+\sigma dB_t,
	\end{equation}
	其中,$\mu,\sigma$为常数,$\{B_t\}_{t \ge 0 }$为一维标准布朗运动,则称$\{X_t\}_{t\ge0}$为算数布朗运动。
\end{Definition}
随机微分方程\ref{BM1}的解为:
$$X_t = X_0 + \mu t+\sigma B_t.$$
\begin{Proof}
	由式$dX_t =\mu dt+\sigma dB_t,$可得
	\begin{align}
		\int_0^t dX_s &  =\int_0^t \mu ds+\int_0^t \sigma dB_s \nonumber\\
		X_t-X_0 & = \mu t+\sigma(B_t-B_0) \nonumber
	\end{align}
	由于$B_0 = 0$,因此整理可得
	$$X_t  = X_0 + \mu t+\sigma B_t .$$
\end{Proof}

\begin{Definition}[几何布朗运动]
	若随机过程$\{X_t\}_{t\ge0}$可以表示为:
	\begin{equation}
		\label{BM2}
		dX_t = \mu X_t dt+\sigma X_t  dB_t,
	\end{equation}
	其中,$\mu,\sigma$为常数,$\{B_t\}_{t \ge 0 }$为一维标准布朗运动,则称$\{X_t\}_{t\ge0}$为几何布朗运动。
\end{Definition}
随机微分方程\ref{BM2}的解为:
$$X_t = X_0\exp\left\{\left(\mu-\frac{1}{2}\sigma^2\right)t+\sigma B_t\right\}.$$
\begin{Proof}
	$$dX_t = \mu X_t dt+\sigma X_t  dB_t$$
	由$It\hat{o}$引理可得
	\begin{align}
		d\ln X_t &= \frac{1}{X_t}dX_t - \frac{1}{2X_t^2}d\langle X \rangle_t \nonumber\\
		& = \mu dt +\sigma dB_t -\frac{\sigma^2}{2} dt \nonumber \\ 
		& = \left(\mu - \frac{\sigma^2}{2} \right) dt + \sigma dB_t \nonumber\\
		\int_0^t d\ln X_s &= \int_0^t \left(\mu - \frac{\sigma^2}{2} \right) ds +\int_0^t \sigma dB_s \nonumber\\
		\ln X_t - \ln X_0 & = \left(\mu - \frac{\sigma^2}{2} \right)t + \sigma (B_t-B_0) \nonumber 
	\end{align}
	由于$B_0 = 0$,因此整理可得
	$$X_t  = X_0 \exp\left\{ \left(\mu - \frac{\sigma^2}{2} \right)t + \sigma B_t\right\}.$$
\end{Proof}
\begin{Theorem}[布朗运动的基本性质]
	设$(B_t)_{t\in R_+}$是一维标准布朗运动,则
	\begin{enumerate}[fullwidth,itemindent=2em] %fullwidth,leftmargin=0pt, 
		\item $(B_t)_{t\in R_+}$是Gauss过程。即对某个指标集$\mathcal{T}$,设$B=\{B_t\}_{t\in\mathcal{T}}$是带流概率空间$(\Omega,\mathcal{F},\mathbb{F},P)$上的一个随机过程,对$\forall (t_1,t_2,\cdots,t_n)\in \mathcal{T},n\ge 1$都有
		$$\left(B_{t_1},B_{t_2},\cdots,B_{t_n}\right)\sim N(\bm{\mu},\Sigma).$$
		其中$\bm{\mu},\Sigma$分别是随机向量$\left(B_{t_1},B_{t_2},\cdots,B_{t_n}\right)$的期望向量和协方差矩阵。
		
		\item 对$\forall t<s$,
		\begin{align}
			P(y,s|x,t):=& P(B_s\le y|B_t = x) = P\left(B_s-B_t+B_t\le y|B_t = x\right)\nonumber\\
			= & P\left(B_s-B_t\le y-x|B_t = x\right)  =  P\left(B_{s-t}\le y-x\right) \nonumber \\
			= & \Phi\left(\frac{y-x}{\sqrt{s-t}}\right). \nonumber 
		\end{align}
		其中,$\Phi$为标准正态分布的分布函数。
	\end{enumerate}
\end{Theorem}

\section{$It\hat{o}$公式和$It\hat{o}$积分}
\begin{Definition}[$It\hat{o}$过程]
	给定带流概率空间$(\Omega,\mathcal{F},\mathbb{F},P)$上的布朗运动$B$,
	假设随机过程$X$具有连续轨道,$X$被称为伊藤过程,如果它具有如下的积分形式:
	\begin{equation}
		\label{eq4}
		X_T= X_0+ \int_{0}^{T}\mu(t)dt  + \int_{0}^{T} \sigma(t)dB_t,\ T\ge0 .
	\end{equation}
	其中,$\mu(t),\sigma(t)$是关于$t$的函数,$B_t$是一维标准布朗运动。
\end{Definition}


伊藤过程$X$的微分形式为:
\begin{equation}
	\label{eq7}
	\mathrm{d}X_t = \mu(t)dt+\sigma(t)\mathrm{d}B_t
\end{equation}


\begin{Theorem}[$It\hat{o}$公式]
	设$X$是具有形式(\ref{eq4})的伊藤过程,$f(t,x)$是一个二元可微实值函数,则过程$f(t,X_t)$仍为$It\hat{o}$过程,且对$\forall t\ge0,$
	$$\mathrm{d}f(t,X_t) = \frac{\partial f}{\partial t}(t,X_t)\mathrm{d}t + \frac{\partial f}{\partial X_t}(t,X_t)\mathrm{d}X_t+\frac{1}{2}\frac{\partial^2 f}{\partial X_t^2}(t,X_t)\mathrm{d} \langle  X \rangle_t. $$
	其中,$\langle X \rangle_t $为$X$的可料二次变差过程(也称尖括号过程)。
	
\end{Theorem}

\begin{Lemma}[$It\hat{o}$引理]
	$$\mathrm{d}X_t = a(t,X_t)\mathrm{d}t+b(t,X_t)\mathrm{d}X_t.$$
	设$f(t,x)$是一个二元可微实值函数,令$f_t:=f(t,X_t)$,则
	\begin{align}
		df_t = & \left[ \frac{\partial f_t}{\partial t}+\frac{1}{2}b^2(X_t,t) \frac{\partial^2 f_t}{\partial X_t^2}\right]dt+\frac{\partial f_t}{\partial X_t}dX_t \nonumber \\
		= &\left[\frac{\partial f_t}{\partial t}+a(t,X_t) \frac{\partial f_t}{\partial X_t}+\frac{1}{2}b^2(t,X_t)) \frac{\partial^2 f_t}{\partial X_t^2} \right]dt +b(t,X_t) \frac{\partial f_t}{\partial X_t}dB_t. \nonumber
	\end{align}
\end{Lemma}

\section{股票价格过程}
假设在风险中性测度下,股票价格$S_t$服从几何布朗运动,即
\begin{equation}
	\label{eq1}
	dS_t = \mu S_t dt+\sigma S_t  dB_t, 
\end{equation}
其中,$\mu$为漂移项(也称为期望回报率),$\sigma$为波动率,$\sigma \ge 0$,且$\mu,\sigma$都为常数。$B_t$服从一维标准布朗运动。

令$f_t = f(S_t,t) = \ln S_t$,则
$$\frac{\partial f_t}{\partial t} = 0,\ \frac{\partial f_t}{\partial S_t} =\frac{1}{S_t},\ 
\frac{\partial^2 f_t}{\partial S_t^2} =-\frac{1}{S_t^2},$$
根据伊藤引理得,
\begin{align}
	\label{eq6}
	d\ln S_t & = df_t  \nonumber \\
	& =  \left(\frac{\partial f_t}{\partial t}+\mu S_t \frac{\partial f_t}{\partial S_t}+
	\frac{1}{2}\sigma^2 S_t^2 \frac{\partial^2 f_t}{\partial S_t^2} \right)dt +\sigma S_t \frac{\partial f_t}{\partial S_t}dB_t \nonumber \\
	& = \left(\mu - \frac{1}{2}\sigma^2 \right)dt+\sigma dB_t .
\end{align}
式(\ref{eq6})与式(\ref{eq7})具有相同的形式,因此,$f_t = \ln S_t$也是一个伊藤过程,则,
\begin{align}
	\label{eq2}
	\ln S_T 
	& = \ln S_0 + \int_{0}^{T}(\mu - \frac{1}{2}\sigma^2)dt + \int_{0}^{T}\sigma dB_t \nonumber \\
	& = \ln S_0 + (\mu-\frac{1}{2}\sigma^2)T + \sigma B_T .
\end{align}
于是,$$S_T = S_0 e^{(\mu-\frac{1}{2}\sigma^2)T + \sigma B_T }.$$
由于$B_T$是标准布朗运动,则
\begin{equation}
	\label{eqJHBM}
	\ln\left(\frac{S_T}{S_0}\right) \sim N\left(\mu-\frac{\sigma^2}{2},\sigma^2T\right).
\end{equation}
将式(\ref{eq2})转换为离散形式,得到
$$\ln S_{t+\delta t}-\ln S_t = \left( \mu - \frac{1}{2}\sigma^2\right)\delta t + \sigma B_{\delta t } .$$

\section{Black-Scholes模型}
Black-Scholes模型假设市场是完备的,即市场满足以下条件:
\begin{enumerate}[itemindent=2em,fullwidth]
	\item[1)] 假设股票价格$S_t$服从几何布朗运动,既满足式(\ref{eq1})的随机微分方程;
	\item[2)] 无风险利率$r$是已知的常数;
	\item[3)] 股票无股息派发;
	\item[4)] 不存在无风险套利机会;
	\item[5)] 对冲投资组合时没有交易价格;
	\item[6)] 股票交易连续,且股票价格的变动也是连续;
	\item[7)] 允许卖空,且股票可分。
\end{enumerate}

假设$V_t = V(S_t,t,E,T)$是一个欧式期权的价值,其中,$t$是期权的起始日期,$T$是期权的终止日期,$E$期权的交割价格,股票$S_t$为期权的标的资产并且无股息派发,$V_t$是一个二元可微实值函数。

在$t$时刻做多一份期权合约$V_t$,同时做空$a$股$S_t$,从而构造组合$\Pi_t$,
$$\Pi_t = V_t - a S_t.$$
由伊藤引理可得,
$$dV_t = \left(\frac{\partial V_t}{\partial t}+\frac{1}{2}\sigma^2S^2_t \frac{\partial^2 V_t}{\partial S_t^2} \right)dt + \frac{\partial V_t}{\partial S_t}dS_t,$$
则,
\begin{align}
	\label{eq3}
	d\Pi_t = & dV_t - adS_t \nonumber \\
	= & \left(\frac{\partial V_t}{\partial t}+\frac{1}{2}\sigma^2S^2_t \frac{\partial^2 V_t}{\partial S_t^2} \right)dt + \left(\frac{\partial V_t}{\partial S_t} - a \right) dS_t.
\end{align}
式(\ref{eq3})可以看作由确定项和随机项组成,随机项为$$\left(\frac{\partial V_t}{\partial S_t} - a \right) dS_t.$$
为了消除随机项,令$$\Delta := a = \frac{\partial V_t}{\partial S_t} .$$这就是$\Delta$对冲,并且通过这种方法构造的组合$\Pi_t$为Delta风险中性的组合。

又由无套利假设可得,
$$d\Pi_t = r\Pi dt.$$
因此可以得到Black-Scholes方程:
$$\frac{\partial V_t}{\partial t} +\frac{ 1}{2}\sigma^2 S_t^2 \frac{\partial^2 V_t}{\partial S_t^2}
+rS_t\frac{\partial V_t}{\partial S_t}-rV_t = 0. $$

\section{二叉树模型}
二叉树模型是期权定价的一种有效方法,它建立在以下假设基础之上:
\begin{enumerate}[itemindent=2em,fullwidth]
	\item[1)] 市场是无摩擦的,即无税、无交易成本,所有资产可无限细分,无卖空限制;
	\item[2)] 投资者和购买者之间无信息差,信息充分共享;
	\item[3)] 允许完全使用卖空得到的资金;
	\item[4)] 允许以无风险利率借入和借出资金;
	\item[5)] 假设股票价格$S_t$服从几何布朗运动,既满足式(\ref{eq1})的随机微分方程;
	\item[6)] 不存在无风险套利的机会。
\end{enumerate}

用$V(S,t_0,K,T)$表示一个标的资产为$S$的期权价格,简记为$V$,$S$无股息派发,期权的生效日期为当前时刻$t_0=0,$到期日为$T$.二叉树模型的基本思想是将时间区间$[0,T]$均分为$n$份,即$0<\frac{T}{n}<\frac{2T}{n}<\cdots<\frac{nT}{n}=T,$讨论每一个$\frac{T}{n}$时间间隔的资产价格的状态。

二叉树模型分为单期二叉树和多期二叉树模型,但是它们的构建思路是基本一致的,接下来对单期二叉树模型进行简要介绍。

当$n=1$时,下图展示了单期二叉树的结构,其中$p_u$表示期初价格为$S$的股票在$T$时刻价格上升到$S_u$的转移概率,反之$p_d$表示期初价格为$S$的股票在$T$时刻价格下降到$S_d$的转移概率。同时,为了保证市场是无套利机会的,$u$和$v$要满足$u>e^{rT}>d$,其中$r$表示无风险利率。
\begin{figure}[H]
	\centering
	\includegraphics[scale=0.77]{OnePeriodBT}
	\caption{单期二叉树模型结构示意图}
\end{figure}

构建一个投资组合,做空一份$V$,做多$\Delta$股$S$,则投资组合的起初价值为:
$$\Phi(0) = -V+\Delta S.$$
在$T$时刻,若股票价格上升到$S_u$,投资组合到期日的价值为:
$$\Phi_u(T)  = -V_u + \Delta S_u.$$
在$T$时刻,若股票价格下降到$S_d$,投资组合到期日的价值为:
$$\Phi_d(T) = -V_d + \Delta S_d.$$
由无套利假设可以得到,不论股票的价格是上涨还是下跌,投资组合的价值是等价的,因此
\begin{align}
	\Phi_u(T) & = \Phi_d(T) \nonumber\\
	 -V_u + \Delta S_u &= -V_d + \Delta S_d \nonumber\\
	 \Delta &= \frac{V_u-V_d}{S_u-S_d}. \nonumber
\end{align}
同样由无套利假设得到$e^{rT}\Phi(0) = \Phi_u(T) = \Phi_d(T).$则
\begin{align}
	e^{rT}\left(-V+\Delta S\right) =& -V_u + \Delta S_u\nonumber\\
	=& -V_u + \frac{V_u-V_d}{S_u-S_d}S_u \nonumber\\
	=& \frac{-S_uV_u+S_dV_u+S_uV_u-S_uV_d}{S_u-S_d}. \nonumber
\end{align}
令$S_u := u\cdot S,\ S_d:=d\cdot S,$则
\begin{align}
	e^{rT}\left(-V+\frac{V_u-V_d}{u-d}\right) =& \frac{dV_u-uV_d}{u-d}.\nonumber
\end{align}
化简,得
\begin{align}
	V =& \frac{V_u-V_d}{u-d} -  e^{-rT}\frac{dV_u-uV_d}{u-d} \nonumber\\
	=& e^{-rT}\left[ \frac{e^{rT}-d}{u-d}V_u+\frac{u-e^{rT}}{u-d}\right] \nonumber \\
	\stackrel{\triangle}{=}& e^{-rT}\left[p_u V_u + p_dV_d \right] .\nonumber
\end{align}
其中,
\begin{align}
p_u &= \frac{e^{rT}-d}{u-d} \nonumber\\
p_d &=1-p_u.\nonumber
\end{align}
因此,只要知道$u$和$d$的值,就可以计算期权在期初的价值。

由于股票价格$S$服从几何布朗运动,既满足式(\ref{eq1})的随机微分方程。则在相同时间间隔下,二叉树模型下股票价格的变化率的一阶矩和二阶矩应与布朗运动下股票价格的变化率的一阶矩和二阶矩相同。

由式(\ref{eqJHBM})可知
\begin{equation*}
	\left\{
	\begin{aligned}
		& p_u u +(1-p_u)d = e^{rT} \nonumber\\
		& pu^2+(1-p)d^2 - e^{2rT}  = e^{2rT} \left(e^{\sigma^2 T}-1\right).
	\end{aligned}
	\right.
\end{equation*}
方程组中包含了3个未知量和2个方程,为了得到解,引入CRR假定:
$$u = \frac{1}{d}.$$
解方程组可以得到,
\begin{equation*}
	\left\{
	\begin{aligned}
		u &= \frac{1}{d} = e^{\sigma\sqrt{T}} \nonumber\\
		p_u & = \frac{e^{rT}-d}{u-d} .\nonumber
	\end{aligned}
	\right.
\end{equation*}
\begin{Proof}
由$p_u u +(1-p_u)d = e^{rT}$可得
$$p_u= \frac{e^{rT}-d}{u-d} .$$
同时将$u=\frac{1}{d}$和$p_u= \frac{e^{rT}-d}{u-d} $带入方程$ pu^2+(1-p)d^2 - e^{2rT}  = e^{rT} \left(e^{\sigma^2 T}-1\right)$中,得
\begin{align}
	\frac{u^3e^{rT}-u^2}{u^2-1}+\frac{u-e^{rT}}{u^3-u} &= e^{2rT+\sigma^2 T} \nonumber\\
	\frac{\left(u^2+1\right)\left(u^2-1\right)e^{rT}-u\left(u^2-1\right)}{u(u^2-1)}&= e^{2rT+\sigma^2 T} \nonumber\\
	\frac{(u^2+1)e^{rT}-u}{u}&= e^{2rT+\sigma^2 T}. \nonumber
\end{align}
整理可得
$$e^{rT}u^2-\left(e^{2rT+\sigma^2T}+1\right)u+e^{rT} = 0 .$$
解方程得到
$$u = \frac{e^{2\left(2r+\sigma^2\right)T}+1+\sqrt{\left(e^{\left(2r+\sigma^2\right)T}+1\right)^2-4e^{2rT}}}{2e^{rT}}.$$
泰勒展开得到
$$u = 1+\sigma\sqrt{T}+\mathcal{O}\left(\sqrt{T}\right).$$
因此可得
$$u = e^{\sigma\sqrt{T}}.$$
以及
$$d = \frac{1}{u} = e^{-\sigma\sqrt{T}}.$$
证毕。
\end{Proof}

\section{隐含波动率}

\subsection{隐含波动率介绍}
隐含波动率是指将根据期权定价模型(如Black-Scholes模型)得到的理论期权价格$V$与期权的市场价格$V_m$建立一个等式关系,来反解得到的波动率。

波动率曲线描述率期权的行权价格和隐含波动率之间的关系,以行权价格为横轴、隐含波动率为纵轴。波动率微笑是指波动率曲线呈现出两边高中间低、开口向上的形状。

在波动率曲线的基础上进一步考虑期限维度,即可得到隐含波动率曲面,隐含波动率曲面可以描述不同执行价格和不同期限结构所对应的隐含波动率。

\subsection{隐含波动率的求解}
本报告在实证分析的部分使用牛顿-拉普森迭代法计算期权的隐含波动率,因此这部分以欧式看涨期权为例对牛顿-拉普森迭代法进行检验介绍。

用$C(S,t,K,T)$表示无股息派发的标的资产为$S$,起始时刻为$t$,执行价格为$K$,终止日期为$T$的欧式看涨期权的理论价格,用$r$表示无风险利率,$\sigma$表示波动率,则有
\begin{equation}
	\label{eq9}
	\left\{
	\begin{aligned}
		&C(S,t,K,T) = SN(d_1)-Ke^{-r(T-t)}N(d_2) \\
		& d_1 = \frac{\ln \frac{S}{K}+\left(r+\frac{\sigma^2}{2}\right)(T-t)}{\sigma\sqrt{T-t}} \\
		& d_2 = d_1 - \sigma\sqrt{T-t} 
	\end{aligned}
	\right. .
\end{equation}
其中,$N(\cdot)$表示标准正态分布的分布函数。

假设$f(\sigma)$是波动率为$\sigma$时的欧式看涨期权的理论价格$C(S,t,K,T)$与欧式看涨期权的市场价格$\hat{C}$之间的差异,即
\begin{align}
	f(\sigma) &= C(S,t,K,T)-\hat{C} \nonumber\\
	& = SN(d_1)-Ke^{-r(T-t)}N(d_2)-\hat{C} .\nonumber
\end{align}
于是可以求得
\begin{align}
	\nu = f^{'}(\sigma) &= SN^{'}(d_1)\sqrt{T-t}  \nonumber\\
	& = \frac{S}{\sqrt{2\pi}}e^{-\frac{d_1^2}{2}}\sqrt{T-t}.\nonumber
\end{align}

利用牛顿-拉普森迭代法计算隐含波动率时,首先需要给波动率赋予一个初值$\sigma_0$(一般通过历史波动率给出一个初值的预测)。通过式(\ref{eq9})计算得到欧式看涨期权的初始理论价格$C_0(S,t,K,T)$,于是得到误差
$$\epsilon_0 = \vert C_0(S,t,K,T) -\hat{C}\vert. $$
设定误差的阈值为$\epsilon$,若$\epsilon_0 \ge \epsilon$,则令
$$\sigma_1  = \sigma_0 - \frac{f(\sigma_0)}{f^{'}(\sigma_0)}.$$
则通过计算得到$C_1(S,t,K,T)$,以及误差
$$\epsilon_1 = \vert C_1(S,t,K,T) -\hat{C}\vert. $$
若$\epsilon_1<\epsilon$,则$\sigma_1$即为所求的隐含波动率;若$\epsilon_1\ge \epsilon$,则继续重复上述步骤,直至误差低于阈值$\epsilon$,对应的波动率即为所求的隐含波动率。