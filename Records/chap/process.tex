\chapter{实证分析}
本报告首先根据已有数据计算2019年12月2日生效、有效期限为一个月的上证50ETF欧式看涨和看跌期权的隐含波动率,得到两类期权在2019年12月2日的期限为一个月的波动率微笑,由于数据的限制,假设不同期限结构的波动率微笑形状相同,然后通过线性插值的方法对未知的行权价格所对应的隐含波动率进行插值处理。

接下来通过式(\ref{eq4})、式(\ref{eq5})和式(\ref{eq8})计算隐含二叉树中各个节点的股价。对每一层枝干,首先计算中心节点的股票价格,再自下而上得到上半部分树节点的股票价格,最后自上而下得到下半部分树节点的股票价格。
\begin{figure}[H]
	\label{ArbFig}
	\centering
	\includegraphics[scale=0.6]{ArbitrageExample}
	\caption{无套利条件示意图}
\end{figure}
在得到每一个节点的股票价格后,都需要检查其是否满足无套利条件,如果不满足无套利条件(即上图中$S_{n+1,i+1}$不满足$F_{n,i}<S_{n+1,i+1}<F_{n,i+1}$这个条件),则将$S_{n+1,i+1}$设置为
$$S_{n+1,i+1} = S_{n+1,i}+\left(S_{n,i}-S_{n,i-1}\right), $$
再继续递推计算剩余节点的股票价格,最终得到一个隐含二叉树。
\begin{figure}[H]
	\centering
	\includegraphics[scale=0.8]{ProcessPath}
	\caption{实证分析流程图}
\end{figure}

\section{求解隐含波动率}
由于上证50EFT期权的合约到期月份为当月、下月及随后两个季月,行权日期为到期月的第四个星期三,因此理论上根据于2019年12月2日生效的期权的数据可以求得6个月的9种不同行权价格对应的隐含波动率。由于要构建24个月的隐含二叉树,根据已有的数据无法得到隐含波动率的期限结构,为简便运算,假设不同期限对应的波动率微笑相同。

2019年12月2日上证50ETF的收盘价为$s = 2.899,$
将$s$以及设置的波动率初始值带入欧式期权的B-S公式得到期权的理论价格,设置收敛阈值为$\epsilon = 1\times 10^{-5}$,及当期权理论价格与市场价格之间的差距小于阈值时,认为算法收敛,此时对应的波动率即为求解得到的隐含波动率,运用牛顿迭代法求2019年12月2日期限为一个月的看涨和看跌期权的隐含波动率。

\begin{figure}[H]
	\begin{minipage}[H]{0.5\linewidth}
		\centering
		\includegraphics[height=4cm,width=7cm]{ImpliedVolatilityCall}
		\caption*{(a) 上证50ETF欧式看涨期权波动率微笑}
	\end{minipage}
	\begin{minipage}[H]{0.5\linewidth}
		\centering
		\includegraphics[height=4cm,width=7cm]{ImpliedVolatilityPut}
		\caption*{(b) 上证50ETF欧式看跌期权波动率微笑}
	\end{minipage}
	\caption{上证50ETF欧式期权波动率微笑}
\end{figure}
而在计算隐含二叉树中各节点的股票价格时,需要用到一些不同上图中已知的9个行权价格所对应的隐含波动率,因此需要估计其他行权价格对应的隐含波动率。为简便运算,对行权价格保留三位小数,在2.7~3.2之间进行线性插值,在小于2.7和大于3.2的部分,假设行权价格每变化0.001,隐含波动率相应变化0.00025.这样得到隐含波动率之后,接下来构建隐含二叉树。


\section{隐含二叉树的构建}
设当前时刻为$t=0$,用$n$表示树的枝干层数,$r$表示年化无风险利率,$T$表示一个月的时间间隔。每一层枝干的节点编号从上至下按照从大到小的顺序进行排列,即第$n$层枝干的节点编号从上至下依次从$n$至1递减。

由于从第一层树生长到第二层树,以及从第二层树生长到第三层树,这两种情况包含了奇数个节点和偶数个节点的情况,因此接下来以这两种情况为例具体介绍计算过程。

\subsection{构建过程}
\subsubsection{1. $\ $当$n=1$时}
第一层枝干也被称为根节点,在这个点上已知上证50ETF的收盘价为$s_0 = 2.899$,以及Arrow-Debreu价格为$1$,同时可以通过公式$$F_1 = e^{rT}s_0$$得到远期合约的价格。
\begin{table}[H]
	\centering
	\caption{根节点已知数据}
	\begin{tabular}{|c|c|c|c|}
		\hline
		 节点编号&股价&Arrow-Debreu价格&远期合约价格 \\\hline
		1&2.899&1&2.9050 \\\hline
	\end{tabular}
\end{table}

\subsubsection{2. $\ $当$n=2$时}

第二层枝干有两个节点,需要根据式(\ref{eq8})来计算这两个节点的股价。
\begin{figure}[H]
	\centering
	\includegraphics[scale=0.8]{Tree1to2}
	\caption{第1层至第2层树的示意图}
\end{figure}
首先从波动率微笑中得到行权价格为$s_1 = 2.899$的欧式看涨期权的隐含波动率,并计算对应的看涨期权价格。
\begin{table}[H]
	\centering
	\caption{行权价格为$s_1 = 2.899$时的隐含波动率及期权价格}
	\begin{tabular}{|c|c|}
		\hline
		看涨期权隐含波动率 & 看涨期权价格$C(s_0,0,s_1,T)$\\\hline
		0.102892&0.037416\\\hline
	\end{tabular}
\end{table}

此时的已知量有:
\begin{table}[H]
	\centering
	\caption{$n=2$时的已知量}
	\begin{tabular}{|c|c|c|}
		\hline
		含义&符号&值 \\\hline
		根节点股价&$s_1$&2.899\\\hline
		远期合约价格&$F_1$&2.9050 \\\hline
		Arrow-Debreu价格&$\lambda_1$&1\\\hline
	\end{tabular}
\end{table}
于是根据公式:
\begin{equation*}
	\left\{
	\begin{aligned}
		& S_2 = \frac{s_1\left[e^{rT}C(s_0,0,s_1,T)+\lambda_1s_1\right]}{\lambda_1F_1-e^{rT}C(s_0,0,s_1,T)}\\
		&S_1 = \frac{s_1^2}{S_2}.
	\end{aligned}
	\right.
\end{equation*}
可以得到
\begin{equation*}
	\left\{
	\begin{aligned}
		& S_2 = 2.969\\ 
		&S_1 = 2.831
	\end{aligned}
	\right. .
\end{equation*}
在得到$S_2$和$S_1$后,通过公式计算出这一期远期合约的价格:
\begin{equation*}
	\left\{
	\begin{aligned}
		& F_{2,2} = e^{rT}S_2 = 2.9752 \\
		&F_{2,1} = e^{rT}S_1 = 2.8369
	\end{aligned}
	\right. ,
\end{equation*}
以及风险中性概率:
$$p_1 = \frac{F_1-S_1}{S_2-S_1},$$
以及这一期的Arrow-Debreu价格:
\begin{equation*}
	\left\{
	\begin{aligned}
		& \lambda_{2,2} = e^{-rT}p_1\lambda_1= 0.5354 \\
		&\lambda_{2,1} = e^{-rT}(1-p_1)\lambda_1 = 0.4624
	\end{aligned}
	\right. .
\end{equation*}
于是,$S_1,S_2,F_{2,1},F_{2,2},\lambda_{2,1},\lambda_{2,2},p_1$成为第二层树到第三层树的已知量。

\subsubsection{3. $\ $当$n=3$时}
用$s_1,s_2$表示上一层得到的股价$S_1,S_2$,而用$S_1,S_2,S_3$表示第三层要求的未知股价,上一层得到的风险中性概率$p_1$在这一层的计算中不再需要使用,因此这一层用$p_2,p_1$表示未知的风险中性概率,其余已知量的符号保持不变。
\begin{figure}[H]
	\centering
	\includegraphics[scale=0.8]{Tree2to3}
	\caption{第2层至第3层树的示意图}
\end{figure}
这一层树的中心节点$S_2$的值应该与期初的股价相同,因此
$$S_2 = s_0 = 2.899.$$

首先考虑上半部分树,从波动率微笑中得到行权价格为$s_2 = 2.969$的欧式看涨期权的隐含波动率,并计算对应的看涨期权价格。
\begin{table}[H]
	\centering
	\caption{行权价格为$s_2 = 2.969$时的隐含波动率及期权价格}
	\begin{tabular}{|c|c|}
		\hline
		看涨期权隐含波动率 & 看涨期权价格$C(s_0,0,s_2,2T)$\\\hline
		0.105505&0.026646\\\hline
	\end{tabular}
\end{table}
于是通过计算可以得到$S_3$的值:
$$S_3 = \frac{S_2\left[e^{rT}C(s_0,0,s_2,2T)\right]-\lambda_{2,2}s_2\left(F_{2,2}-S_2\right)}{e^{rT}C(s_0,0,s_2,2T)-\lambda_{2,2}\left(F_{2,2}-S_2\right)} = 3.102 \ .$$

再考虑下半部分树,从波动率微笑中得到行权价格为$s_1 = 2.831$的欧式看跌期权的隐含波动率,并计算对应的看跌期权价格。
\begin{table}[H]
	\centering
	\caption{行权价格为$s_1 = 2.831$时的隐含波动率及期权价格}
	\begin{tabular}{|c|c|}
		\hline
		看跌期权隐含波动率 & 看跌期权价格$P(s_0,0,s_1,2T)$\\\hline
		0.115473&0.023030\\\hline
	\end{tabular}
\end{table}
于是通过计算可以得到$S_1$的值:
$$S_1 = \frac{S_2\left[e^{rT}P(s_0,0,s_1,2T)\right]+	\lambda_{2,1}s_1\left(F_{2,1}-S_2\right)}{e^{rT}P(s_0,0,s_1,2T)+
	\lambda_{2,1}\left(F_{2,1}-S_2\right)} = 2.553\ .$$

在得到$S_3$、$S_2$和$S_1$后,通过公式计算出这一期远期合约的价格:
\begin{equation*}
	\left\{
	\begin{aligned}
		& F_{3,3} = e^{rT}S_3 = 3.1085 \\
		&F_{3,2} = e^{rT}S_2 = 2.9050 \\
		& F_{3,1} = e^{rT}S_1 = 2.5583
	\end{aligned}
	\right. ,
\end{equation*}
以及风险中性概率:
\begin{equation*}
	\left\{
	\begin{aligned}
		& p_2 = \frac{F_{2,2}-S_2}{S_3-S_2} = 0.3753\\
		& p_1 = \frac{F_{2,1}-S_1}{S_2-S_1} = 0.8205
	\end{aligned}
	\right. ,
\end{equation*}
以及这一期的Arrow-Debreu价格:
\begin{equation*}
	\left\{
	\begin{aligned}
		&\lambda_{3,3} = e^{-rT}p_2\lambda_{2,2} = 0.2006\\
		& \lambda_{3,2} = e^{-rT}(p_1\lambda_{2,1}+(1-p_2)\lambda_{2,2})= 0.7125 \\
		&\lambda_{3,1} = e^{-rT}(1-p_1)\lambda_{2,1} = 0.0828
	\end{aligned}
	\right. .
\end{equation*}
因此,$S_1,S_2,S_3,F_{3,1},F_{3,2},F_{3,3},\lambda_{3,1},\lambda_{3,2},\lambda_{3,3},p_1,p_2$成为第三层树到第四层树的已知量。

按照上述步骤依次计算每一层树的节点股价即可,但是在计算出每一个节点股价之后,要对其进行无套利条件的检验。如果不满足无套利条件(即图(\ref{ArbFig})中$S_{n+1,i+1}$不满足$F_{n,i}<S_{n+1,i+1}<F_{n,i+1}$这个条件),则将$S_{n+1,i+1}$设置为
$$S_{n+1,i+1} = S_{n+1,i}+\left(S_{n,i}-S_{n,i-1}\right), $$
再继续递推计算剩余节点的股票价格,最终构建出隐含二叉树。


\subsection{构建结果}
由于篇幅限制,报告中只画出前12层树,完整的隐含二叉树的数据将于附件展示。
\begin{figure}[H]
	\centering
	\includegraphics[scale=0.3]{12NodesTree}
	\caption{得到的股票价格二叉树的一部分}
\end{figure}
可以看到隐含二叉树与一般的二叉树的结构有所不同,前者一般不会出现完全对称的结构,而后者的结构是完全对称的。
\begin{figure}[H]
	\centering
	\includegraphics[scale=0.5]{ArrowDeTree}
	\caption{得到的Arrow-Debreu树的一部分}
\end{figure}
\begin{figure}[H]
	\centering
	\includegraphics[scale=0.5]{ForwardPriceTree}
	\caption{得到的远期合约价格树的一部分}
\end{figure}
\begin{figure}[H]
	\centering
	\includegraphics[scale=0.5]{ProbTree}
	\caption{得到的风险中性概率树的一部分}
\end{figure}
二叉树中每一步的转移概率是一致的,但是通过得到的数据可以看到,隐含二叉树的风险中性概率并没有这个特点。

\subsection{隐含二叉树的作用}
传统的Black-Scholes模型假设波动率是常数,隐含二叉树是对Black-Scholes模型的一个拓展,它引入从市场中得到的隐含波动率,使得构建出来的树与市场之间具有一致性。

由于隐含波动率是市场期权价格的一个内含值,而隐含二叉树的实质是根据隐含波动率计算得到的股票未来价格的离散分布,因此得到的股票价格的分布是在风险中性概率测度下的分布。而在得到了股票价格未来的在风险中性概率测度下的分布之后,有以下几种作用:
\begin{enumerate}
	\item [1)] 不仅可以都欧式期权进行定价,还可以对其他类型的期权,如:美式期权、障碍期权、其他路径依赖型期权进行定价,可以应用的范围比较广;
	\item [2)]可以与蒙特卡洛模拟方法进行结合,通过随机模拟生成股票价格的变化路径,分别对每一条路径进行期权支付规则的分析,得到每一条路径的收益现值,对所有路径的现值取平均即可得到期权的价格。
\end{enumerate}
