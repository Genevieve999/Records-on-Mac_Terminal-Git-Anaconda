\chapter{隐含二叉树}
本章首先对构建隐含二叉树所需要的Arrow-Debreu价格进行简要介绍。在构建隐含二叉树的过程中,将隐含二叉树以中心节点为界,分为上半部分树和下半部分树两个部分。因此接下来首先分别讨论当枝干层节点数为奇数和偶数时中心节点的股票价格公式,然后分别给出上半部分树和下半部分树的计算公式,最后讨论当得到的股票价格不满足无套利条件时合适的处理方法。

\section{Arrow-Debreu价格}
假设起始时刻$t = 0$,构造二叉树时每一步的步长为$\Delta t$,则第$n$个节点所处的时刻为$t_n = n\Delta t$.用$S_{n,i}$表示标的资产在第$n$个节点处于状态$i$时的价格,用$(n,i)$表示标的资产所处的位置,用$p_{n,i}$表示标的资产从$(n,i)$转移到$(n+1,i+1)$的转移概率,用$r$表示无风险收益率。

则从第$n$步到第$n+1$步的二叉树结构如下:
\begin{figure}[H]
	\centering
	\includegraphics[scale=0.75]{Tree1}
	\caption{第$n$步到第$n+1$步的二叉树结构示意图}
\end{figure}

假设存在一个证券,该证券的支付规则为:若标的资产处于状态$i$,则支付1元;若标的资产处于其他状态,则支付0元。用$\lambda_{n,i}$表示该证券在时刻$n$的价格,
$$	\lambda_{0,0} = 1,$$
\begin{equation*}
	\lambda_{n+1,i} = \left\{ 
	\begin{aligned}
		& e^{-r\Delta t} \left(p_{n,n}\lambda_{n,n}\right),  \qquad i = n+1\nonumber \\
		& e^{-r\Delta t}\left(p_{n,i-1}\lambda_{n,i-1}+\left(1-p_n,i\right)\lambda_{n,i}\right) ,\qquad 1\le  i\le  n \nonumber \\
		& e^{-r\Delta t} \left(1-p_{n,0}\right)\lambda_{n,0} ,\qquad i=0 \nonumber
	\end{aligned} \right.
\end{equation*}
则$\lambda_{n,i}$就被称为Arrow-Debreu价格。

以一个步长为$\Delta t$的两步二叉树为例。
\begin{figure}[H]
	\centering
	\includegraphics[scale=0.75]{ArrowDebreuTree}
	\caption{Arrow-Debreu价格的两步二叉树示例}
\end{figure}
则可以计算得到各个节点上的Arrow-Debreu价格:
\begin{align}
	\lambda_{0,0} & = 1, \nonumber\\
	\lambda_{1,1} & = e^{-r\Delta t} p_{0,0}\lambda_{0,0}, \nonumber\\
	\lambda_{1,0} & = e^{-r\Delta t} (1-p_{0,0})\lambda_{0,0} ,\nonumber \\
	\lambda_{2,2} & =  e^{-r\Delta t} p_{1,1}\lambda_{1,1},\nonumber\\
	\lambda_{2,1} & =  e^{-r\Delta t} \left[p_{1,0}\lambda_{1,0}+(1-p_{1,1})\lambda_{1,1}\right],\nonumber\\
	\lambda_{2,0}& = e^{-r\Delta t} (1-p_{1,0})\lambda_{1,0}. \nonumber
\end{align}

\section{中心节点}
假设起始时刻$t = 0$,对应树的第一层,也称为根节点,对应的股票价格为$s_{0,0}$。构造二叉树时每一步的步长为$\Delta t$,则第$n$层枝干所处的时刻为$t_n = (n-1)\Delta t$.假设已经构造了前$n$步二叉树节点。

\subsection{当$n$为偶数时}
\begin{figure}[H]
	\centering
	\includegraphics[scale=0.7]{OddCenter}
	\caption{$n$为偶数时隐含二叉树中心节点情况}
\end{figure}
此时,中心节点$S_{\frac{n+1}{2}}$的取值为:
$$S_{\frac{n+1}{2}} = s_{0,0}.$$
\subsection{当$n$为奇数时}
\begin{figure}[H]
	\centering
	\includegraphics[scale=0.7]{EvenCenter}
	\caption{$n$为奇数时隐含二叉树中心节点情况}
\end{figure}
此时$s_{\frac{n+1}{2}} = s_{0,0}$,而第$n+1$步的中心需要在CRR假设下进行计算,具体计算方法在后续介绍。

\section{其他节点}
构建隐含二叉树时,需要根据中心节点的值对上下两部分树的节点值进行递推,当$n$为偶数时,由上面的讨论可以知道第$n+1$层中心点的值与初值相同,讨论起来比较直观,因此首先对$n$为偶数时的情况进行讨论,再对$n$为奇数时的情况进行讨论。
\subsection{当$n$为偶数时}
同样假设起始时刻$t = 0$,构造二叉树时每一步的步长为$\Delta t$,则第$n$个节点所处的时刻为$t_n = n\Delta t$.

假设已经构造了前$n$步二叉树节点,$n$为奇数,用$s_i(1\le i\le n)$表示已经通过计算得到的第$n$步处于状态$i$的标的资产的价格,用$S_i(1\le i\le n+1)$表示未知的第$n+1$步处于状态$i$的标的资产的价格,用$\lambda_i(1\le i\le n)$表示已经通过计算得到的第$n$步处于状态$i$的Arrow-Debreu价格,用$F_i = e^{r\Delta t}s_i$表示标的资产$s_i$在$t_{n+1}$时刻的远期合约价格,用$p_i(1\le i\le n)$表示未知的从节点$(n,i)$转移到节点$(n+1,i+1)$的转移概率。
\begin{table}[H]
	\centering
	\caption{符号及其含义}
	\begin{tabular}{|l|l|}
		\hline
		符号 & 含义 \\\hline
		$r$ & 已知的无风险利率 \\\hline
		$s_{0,0}$ & 已知的起初时刻标的资产价格 \\\hline
		$s_i,\quad 1\le i \le n$ & 已知的处于节点$(n,i)$的标的资产价格 \\\hline
		$F_i,\quad 1\le i \le n$ & 已知的处于节点$(n,i)$的远期合约价格 \\\hline
		$\lambda_i,\quad 1\le i \le n$ & 已知的处于节点$(n,i)$的Arrow-Debreu价格 \\\hline
		$S_i,\quad 1\le i \le n+1$ & 未知的处于节点$(n+1,i)$的标的资产价格  \\\hline
		$p_i,\quad 1\le i \le n$ & 未知的从节点$(n,i)$转移到节点$(n+1,i+1)$的转移概率 \\\hline
	\end{tabular}
\end{table}
当$n$为奇数时,树的第$n+1$步的中心与标的资产起初的价格应该在同一水平线上。第$n+1$步的树上有$2n+1$个未知数:分别是$n$个转移概率和$n+1$个标的资产的价格状态。需要通过解方程组来得到这$2n+1$个量的值。

用$C(S,t_0,s_i,t_{n+1})$和$P(S,t_0,s_i,t_{n+1})$分别表示起始日期为当前时刻$t_0$,标的资产为$S$,行权价格为$s_i(1\le i \le n)$,到期日为$t_{n+1}$的欧式看涨期权和欧式看跌期权的价格。
\begin{figure}[H]
	\centering
	\includegraphics[scale=0.7]{OddTree}
	\caption{$n$为偶数时的隐含二叉树结构}
\end{figure}

\subsubsection{1. $\ $上半部分树}
对于上半部分树,可列方程组
\begin{equation*}
	\left\{
	\begin{aligned}
		& F_i = p_iS_{i+1}+(1-p_i)S_i \nonumber\\
		& C(S,t_0,s_i,t_{n+1}) = e^{-r \Delta t}\sum_{j=1}^{n} \left[\lambda_jp_j+\lambda_{j+1}(1-p_{j+1})\right]\max(S_{j+1}-s_i,0).\nonumber \\
		& \lambda_{n+1} = p_{n+1}=0 \nonumber
	\end{aligned}
	\right.
\end{equation*}
得到解为
\begin{equation}
	\label{eq4}
	\left\{
	\begin{aligned}
		& p_i = \frac{F_i-S_i}{S_{i+1}-S_i }\\
		& S_{i+1} = \frac{S_i\left[e^{r\Delta t}C(S,t_0,s_i,t_{n+1}) -\Sigma \right]-\lambda_is_i(F_i-S_i)}{\left[e^{r\Delta t}C(S,t_0,s_i,t_{n+1}) -\Sigma \right]-\lambda_i(F_i-S_i)}.
	\end{aligned}
	\right.
\end{equation}
其中,
$$ \Sigma: = \sum_{j=i+1}^n \lambda_j (F_j-s_i).$$
\begin{Proof}
	由$F_i = p_iS_{i+1}+(1-p_i)S_i$可得
	$$p_i = \frac{F_i-S_i}{S_{i+1}-S_i }.$$
	化简方程$C(S,t_0,s_i,t_{n+1}) = e^{-r \Delta t}\sum_{j=1}^{n} \left[\lambda_jp_j+\lambda_{j+1}(1-p_{j+1})\right]\max(S_{j+1}-s_i,0),$
	\begin{align}
		e^{r\Delta t}C(S,t_0,s_i,t_{n+1}) 
		 = & \sum_{j=i}^{n} \left[\lambda_jp_j+\lambda_{j+1}(1-p_{j+1})\right]\max(S_{j+1}-s_i,0) \nonumber\\
		 = & \left[\lambda_i p_i +\lambda_{i+1}(1-p_{i+1})\right] (S_{i+1}-s_i) \nonumber\\
		     & + \left[\lambda_{i+1} p_{i+1} +\lambda_{i+2}(1-p_{i+2})\right] (S_{i+2}-s_i) \nonumber\\
		     & + \cdots \nonumber\\
		     & + \lambda_n p_n (S_{n+1}-s_i) \nonumber\\
		= & \lambda_i p_i(S_{i+1}-s_i) \nonumber\\
		   & + \lambda_{i+1}\left\{ \left[(1-p_{i+1})S_{i+1}+p_{i+1}S_{i+2}\right]-s_i\right\} \nonumber\\
		   & + \cdots \nonumber\\
		   & + \lambda_n \left\{  \left[(1-p_{n})S_{n}+p_{n}S_{n+1}\right]-s_i\right\} \nonumber\\
		= & \lambda_i p_i(S_{i+1}-s_i) \nonumber\\
		& +  \lambda_{i+1}\left[F_{i+1}-s_i\right]\nonumber\\
		& + \cdots \nonumber\\
		& + \lambda_n \left[F_n-s_i\right]\nonumber \\
		= & \lambda_i p_i(S_{i+1}-s_i)+\Sigma. \nonumber
	\end{align}
	则
	\begin{align}
		S_{i+1}-s_i =& \frac{e^{r\Delta t}C(S,t_0,s_i,t_{n+1}) -\Sigma}{\lambda_i p_i} \nonumber \\
		=& \frac{e^{r\Delta t}C(S,t_0,s_i,t_{n+1}) -\Sigma}{\lambda_i (F_i-S_i)}(S_{i+1}-S_i). \nonumber
	\end{align}
	进一步化简得
	$$\frac{e^{r\Delta t}C(S,t_0,s_i,t_{n+1}) -\Sigma-\lambda_i(F_i-S_i)}{\lambda_i (F_i-S_i)} S_{i+1}
	= \frac{S_i\left[e^{r\Delta t}C(S,t_0,s_i,t_{n+1}) -\Sigma\right]-\lambda_is_i(F_i-S_i)}{\lambda_i (F_i-S_i)}.$$
	因此,
	$$S_{i+1} = \frac{S_i\left[e^{r\Delta t}C(S,t_0,s_i,t_{n+1}) -\Sigma \right]-\lambda_is_i(F_i-S_i)}{\left[e^{r\Delta t}C(S,t_0,s_i,t_{n+1}) -\Sigma \right]-\lambda_i(F_i-S_i)}.$$
	证毕。
\end{Proof}

\subsubsection{2. $\ $下半部分树}
对于下半部分树,可列方程组
\begin{equation*}
	\left\{
	\begin{aligned}
		& F_i = p_iS_{i+1}+(1-p_i)S_i \\
		& P(S,t_0,s_i,t_{n+1}) = e^{-r \Delta t}\sum_{j=0}^{n} \left[\lambda_jp_j+\lambda_{j+1}(1-p_{j+1})\right]\max(s_i-S_{j+1},0).\nonumber\\
		& \lambda_0 = p_0 = 0 \nonumber
	\end{aligned}
	\right.
\end{equation*}
得到解为:
\begin{equation}
	\label{eq5}
	\left\{
	\begin{aligned}
		& p_i = \frac{F_i-S_i}{S_{i+1}-S_i }\\
		& S_{i} = \frac{S_{i+1}\left[e^{r\Delta t}P(S,t_0,s_i,t_{n+1}) -\Sigma_1 \right]+\lambda_is_i(F_i-S_{i+1})}{\left[e^{r\Delta t}P(S,t_0,s_i,t_{n+1}) -\Sigma_1 \right]+\lambda_i(F_i-S_{i+1})}.
	\end{aligned}
	\right.
\end{equation}
其中,$$\Sigma_1 = \sum_{j=1}^{i-1}\lambda_j(s_i-F_j).$$
\begin{Proof}
	由$F_i = p_iS_{i+1}+(1-p_i)S_i$可得
	$$p_i = \frac{F_i-S_i}{S_{i+1}-S_i }.$$
	则
	$$1-p_i = \frac{S_{i+1}-F_i}{S_{i+1}-S_i}.$$
	\begin{align}
		e^{r\Delta t}P(S,t_0,s_i,t_{n+1}) =& \sum_{j=0}^n \left[\lambda_jp_j+\lambda_{j+1}(1-p_{j+1})\right]\max(s_i-S_{j+1},0) \nonumber\\
		=&\lambda_1(1-p_1)(s_i-S_1) \nonumber\\
		& + \left[\lambda_1p_1+\lambda_2(1-p2)\right](s_i-S_2) \nonumber\\
		& + \left[\lambda_2p_2+\lambda_3(1-p3)\right](s_i-S_3) \nonumber\\
		& + \cdots \nonumber\\
		& + \left[\lambda_{i-1}p_{i-1}+\lambda_i(1-p_i)\right](s_i-S_i) \nonumber\\
		=& \lambda_1 (s_i-F_1) + \lambda_2 (s_i-F_2)+\cdots \nonumber\\
		& + \lambda_{i-1}(s_i-F_{i-1}) + \lambda_i(1-p_i)(s_i-S_i) \nonumber\\
		=&  \lambda_i(1-p_i)(s_i-S_i) + \sum_{j=1}^{i-1}\lambda_j(s_i-F_j) \nonumber \\
		=& \lambda_i(1-p_i)(s_i-S_i) +\Sigma_1. \nonumber
	\end{align}
	则,
	\begin{align}
		s_i-S_i &= \frac{e^{r\Delta t}P(S,t_0,s_i,t_{n+1})-\Sigma_1}{\lambda_i(1-p_i)} \nonumber\\
						&=\frac{e^{r\Delta t}P(S,t_0,s_i,t_{n+1})-\Sigma_1}{\lambda_i(S_{i+1}-F_i)} (S_{i+1}-S_i).\nonumber
	\end{align}
	进一步化简得
	$$\frac{e^{r\Delta t}P(S,t_0,s_i,t_{n+1})-\Sigma_1-\lambda_i(S_{i+1}-F_i)}{\lambda_i(S_{i+1}-F_i)}S_i = 
	\frac{S_{i+1}\left[e^{r\Delta t}P(S,t_0,s_i,t_{n+1})-\Sigma_1\right]-\lambda_is_i(S_{i+1}-F_i)}{\lambda_i(S_{i+1}-F_i)}.$$
	因此,
	$$S_{i} = \frac{S_{i+1}\left[e^{r\Delta t}P(S,t_0,s_i,t_{n+1}) -\Sigma_1 \right]+\lambda_is_i(F_i-S_{i+1})}{\left[e^{r\Delta t}P(S,t_0,s_i,t_{n+1}) -\Sigma_1 \right]+\lambda_i(F_i-S_{i+1})}.$$
	证毕。
\end{Proof}
因此可以通过式(\ref{eq4})和(\ref{eq5})计算每个节点的$S_{j}$和转移概率$p_j$.

\subsection{当$n$为奇数时}
当第$n$层枝干的节点数为奇数时,其中心为$s_{\frac{n+1}{2}} = s_{0,0}.$
\begin{figure}[H]
	\centering
	\includegraphics[scale=0.7]{EvenTree}
	\caption{$n$为奇数时的隐含二叉树结构}
\end{figure}
引入CRR二叉树的假设,即假设标的资产价格$S$在间隔$\Delta t$之后向上转移到$u\times S$,向下转移到$d\times S$,其中$u,d$满足的条件为
\begin{equation*}
	\left\{
	\begin{aligned}
		& u = \frac{1}{d}, \nonumber\\
		& u > d. \nonumber
	\end{aligned}
	\right.
\end{equation*}
利用CRR二叉树的假设来确定$S_{\frac{n+1}{2}}$和$S_{\frac{n+3}{2}}$,即
$$S_{\frac{n+1}{2}} \times S_{\frac{n+3}{2}} = s_{0,0}^2.$$
可以得到
\begin{equation}
	\label{eq8}
	\left\{
	\begin{aligned}
		p_i &= \frac{F_i-S_i}{S_{i+1}-S_i } \\
		S_{\frac{n+3}{2}} & = \frac{s_{0,0}\left[e^{r\Delta t}C(S,t_0,s_{0,0},t_{n+1})+\lambda_{\frac{n+1}{2}} s_{0,0} - \Sigma\right]}{\lambda_{\frac{n+1}{2}}F_{\frac{n+1}{2}}-e^{r\Delta t}C(S,t_0,s_{0,0},t_{n+1})+\Sigma} \\
		S_{\frac{n+1}{2}} & = \frac{s_{0,0}\left(\lambda_{\frac{n+1}{2}} F_{\frac{n+1}{2}} - e^{r\Delta t}C(S,t_0,s_{0,0},t_{n+1}) + \Sigma \right)}{e^{r\Delta t}C(S,t_0,s_{0,0},t_{n+1})+\lambda_{\frac{n+1}{2}} s_{0,0}-\Sigma}.
	\end{aligned}
	\right.
\end{equation}
其中,
$$ \Sigma: = \sum_{j={\frac{n+3}{2}}}^n \lambda_j (F_j-s_i).$$
\begin{Proof}
	由式(\ref{eq4})可得
	$$S_{i+1}\left[e^{r\Delta t}C(S,t_0,s_{0,0},t_{n+1}) -\Sigma \right]-\lambda_iS_{i+1}(F_i-S_i) = S_i\left[e^{r\Delta t}C(S,t_0,s_{0,0},t_{n+1}) -\Sigma \right]-\lambda_is_i(F_i-S_i).$$
	其中,$i = \frac{n+1}{2},\ i+1 = \frac{n+3}{2}$
	又由于
	$$e^{r\Delta t}C(S,t_0,s_{0,0},t_{n+1})  = \lambda_ip_i(S_{i+1}-s_{0,0})+\Sigma,$$
	$$F_i = p_iS_{i+1}+(1-p_i)S_i,$$
	可以得到
	\begin{align}
		S_{i+1}\left[e^{r\Delta t}C(S,t_0,s_{0,0},t_{n+1}) -\Sigma \right]-\lambda_iS_{i+1}F_i 
		=& -\lambda_is_{0,0}^2+S_i\left(\lambda_ip_i(S_{i+1}-s_{0,0})\right)-\lambda_is_i(F_i-S_i) \nonumber\\
		=&  -\lambda_is_{0,0}^2 \nonumber\\
		&+ \lambda_ip_is_{0,0}^2-\lambda_ip_iS_is_{0,0}\nonumber\\
		&-\lambda_ip_iS_{i+1}s_{0,0}+\lambda_ip_iS_is_{0,0}\nonumber\\
		= & -\left[\lambda_is_{0,0}^2+\lambda_ip_is_{0,0}(S_{i+1}-s_{0,0})\right] \nonumber\\
		= & -\left[\lambda_is_{0,0}^2+s_{0,0}\left(e^{r\Delta t}C(S,t_0,s_{0,0},t_{n+1})-\Sigma\right)\right] .\nonumber 
	\end{align}
	将$i = \frac{n+1}{2},\ i+1 = \frac{n+3}{2}$代入,整理可得
	$$S_{\frac{n+3}{2}} = \frac{s_{0,0}\left[e^{r\Delta t}C(S,t_0,s_{0,0},t_{n+1})+\lambda_{\frac{n+1}{2}} s_{0,0} - \Sigma\right]}{\lambda_{\frac{n+1}{2}} F_{\frac{n+1}{2}}-e^{r\Delta t}C(S,t_0,s_{0,0},t_{n+1})+\Sigma}.$$
	最后,由$S_{\frac{n+3}{2}}\times S_{\frac{n+1}{2}} = s_{0,0}^2$可得
	$$S_{\frac{n+1}{2}}  = \frac{s_{0,0}\left(\lambda_{\frac{n+1}{2}}  F_{\frac{n+1}{2}} - e^{r\Delta t}C(S,t_0,s_{0,0},t_{n+1}) + \Sigma \right)}{e^{r\Delta t}C(S,t_0,s_{0,0},t_{n+1})+\lambda_{\frac{n+1}{2}} s_{0,0}-\Sigma}.$$
	证毕。
\end{Proof}
在得到树中心的两个资产价格的初始化价格后,可分别通过式(\ref{eq4})和式(\ref{eq5})求上半部分树和下半部分树其他节点上的标的资产价格状态和转移概率。

\section{无套利条件}
\begin{figure}[H]
	\centering
	\includegraphics[scale=0.6]{ArbitrageExample}
	\caption{无套利条件示意图}
\end{figure}
由无套利假设可得每个节点上的转移概率都必须在$[0,1]$内。若从节点$(n,i)$到节点$(n+1,i+1)$转移概率大于1,则上半部分树的节点$(n+1,i+1)$上的价格$S_{n+1,i+1}<F_{n,i}.$若从节点$(n,i)$到节点$(n+1,i)$转移概率小于0,则下半部分树的节点$(n+1,i)$上的价格$S_{n+1,i}>F_{n,i}.$这两种情况都存在无风险套利的机会。因此必须要求计算得到的股票价格落在对应的两个远期价格之间,即
$$F_{n,i}<S_{n+1,i+1}<F_{n,i+1}.$$

若根据隐含波动率计算出来的股票价格不满足无套利条件,就要推翻通过隐含波动率得到的期权价格所计算得到的节点的股票价格,并将股票价格设定在上述不等式区间内,本报告采取的方法是使$S_{n+1,i+1}$与$S_{n+1,i}$之间的距离与$S_{n,i}$与$S_{n,i-1}$之间的距离保持一致,即将$S_{n+1,i+1}$设置为
$$S_{n+1,i+1} = S_{n+1,i}+\left(S_{n,i}-S_{n,i-1}\right).$$
