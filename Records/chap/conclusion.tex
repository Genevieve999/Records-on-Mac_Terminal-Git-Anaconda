\chapter{小结}
% 页面格式
\thispagestyle{fancy}

本报告从理论出发,依次介绍期权定价相关理论、一般二叉树模型和隐含二叉树模型,并在实证分析部分以2019年12月2日为起始日期,以一个月为间隔时间单位,对上证50ETF构造了24层的隐含二叉树。

本报告首先假设波动率微笑在这两年的结构是一致的,然后利用牛顿迭代法求得上证50ETF欧式看涨和看跌期权的隐含波动率,并对不同行权价格对应的隐含波动率进行线性插值。

首先假设隐含波动率的期限结构不变以及用线性插值的方法估计隐含波动率的好处是简便运算,但是隐含波动率的期限结构不变是与市场不符的,这会导致得到的股票价格分布与风险中性测度下的股票价格分布有一定的差异,从而进一步导致计算期权价格时的不准确。因此应该对市场的数据进行隐含波动率曲面进行建模,将波动率的偏度和期限结构同时考虑到模型内,这样才能与市场信息更为相符。

其次直接对波动率微笑进行线性插值是比较死板的方法,并没有考虑在已知的行权价格之间隐含波动率其他的变化方式。

在得到了波动率微笑之后,本报告构建了24层的隐含二叉树,以第一层到第三层为例介绍了分别介绍了奇数节点层和偶数节点层中心节点价格的计算,并分别对上半部分树和下半部分树进行构建,给出了各个节点构造的详细过程,最后展示了树的部分结果。