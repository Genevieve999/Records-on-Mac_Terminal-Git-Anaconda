\begin{center}
    \thispagestyle{empty}
    \fontsize{18pt}{\baselineskip}\bf\textsf{摘\quad 要}
    \vspace{10pt}
    
\end{center}

\vspace{10pt}

本报告选择2019年12月2日作为起始日期,根据上证50ETF欧式看涨期权和上证50ETF欧式看跌期权的市场历史数据计算出不同行权价格对应的隐含波动率,画出波动率微笑曲线,并通过线性插值的方法得到不同行权价格对应的隐含波动率。

本报告的主题是隐含二叉树的构建。第一步从理论出发,在构建隐含二叉树的过程中,以中心节点为界,将每一层枝干分为上半部分树和下半部分树。首先分别讨论了节点数为奇数和偶数两种情况下中心节点股票价格的计算方法,接下来分别对上半部分树和下半部分树各节点上的股票价格计算公式进行讨论,并证明了股票价格的计算公式,同时讨论了当计算得到的股票价格不满足无套利条件时对其的处理方法。第二步从实证出发,根据得到的隐含波动率,对2019年12月2日的上证50ETF收盘价构建隐含二叉树并简要讨论了隐含二叉树的作用。



\begin{figure}[h]
    \textbf{关键词:} 隐含波动率,隐含二叉树
\end{figure}